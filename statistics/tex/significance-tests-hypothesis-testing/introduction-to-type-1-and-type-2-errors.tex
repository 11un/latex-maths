% 0. Preamble: -----------------------------------------------------{{{
% 0.1 Document Class Setup: ----------------------------------------{{{

\documentclass[12pt]{article}

%}}}
% 0.2 Packages + Libraries: ----------------------------------------{{{

\usepackage{amsfonts, array, amsmath, amssymb, bm, caption, booktabs, mathtools, tikz}
\usepackage[margin=0.21in]{geometry}
\usepackage[]{microtype}
\usetikzlibrary{shapes.geometric,arrows,positioning,fit, calc}

%}}}
% 0.3 Commands: ----------------------------------------------------{{{

% binomial exponents - 3 args, 1st arg 2
\newcommand{\negBi}[3][2]{{(#2- #3)}^{#1}}
\newcommand{\negBiBold}[3][2]{\bm{{(#2- #3)}^{#1}} }

%}}}
%}}}
\begin{document}
% Section: Introduction to Type I and Type II Errors ---------------{{{
\section*{Introduction to Type I and Type II Errors:}
% Subsection: Significance Test ------------------------------{{{
\subsection*{Significance Test:}
Define a null and alternative hypothesis for a population in question. Using a sample of the population, calculate a statistic that can estimate something about the population. (The parameter in question.)
\begin{align}
  H_{0}: \text{Null Hypothesis (status quo)} &&
  H_{a}: \text{Alternative Hypothesis (something different)}
\end{align}

\begin{center}
\begin{tikzpicture}
  \draw [fill=white] (0,0) circle (3.0cm)
      node (population) {};
  \node (poplabel) [above = 1.75 of population]{Population};
  \draw [dashed] (1.5,0.75) circle (0.9cm)
      node (sample) {};
  % Using tikzpicture positioning library
    \node (samplabel) [above right = 2.75 of sample]{(Sample) statistic};
  % Draw arrow
  \draw[->, thick, to path={-- (\tikztotarget)}]
    (samplabel) edge (sample);
\end{tikzpicture}
\end{center}
%}}}
% Subsection: p-value -------------------------------------{{{
\subsection*{p-value}
Using the statistic, find the probability of getting that statistic, given the null hypothesis is true.

\begin{equation}
  \text{p-value} = P(statistic \mid H_{0} \; \text{true})
\end{equation}

If the p-value is less than the significance level, then the null hypothesis is rejected:
\begin{equation}
  \text{p-value}< \alpha \rightarrow \text{reject}\; H_{0}
\end{equation}

If the p-value is greater than or equal to the significance level, then we fail to reject the null hypothesis:

\begin{equation}
  \text{p-value}\ge \alpha \rightarrow \text{reject}\; H_{0}
\end{equation}
%}}}
% Subsection: Understanding the Type Errors ----------------------------{{{
\subsection*{Understanding the Type Errors}
\begin{itemize}
  \item When the null hypothesis is true, if it is rejected, that is a type 1 error.
  \item When the null hypothesis is false, if it is failed to be rejected that is a type 2 error.
\end{itemize}
\begin{table}[h!]
\begin{center}
  \begin{tabular}{l c c}
    & \textbf{$\bm{H_{0}}$ true} & \textbf{$\bm{H_{0}}$ false}\\
    \midrule \\
    \textbf{Reject $\bm{H_{0}}$} & Type I Error & Correct \\ \\
    \textbf{Fail to Reject $\bm{H_{0}}$} & Correct & Type II Error\\ \\
    \bottomrule
  \end{tabular}
\end{center}
\end{table}
%}}}
%}}}
\end{document}
