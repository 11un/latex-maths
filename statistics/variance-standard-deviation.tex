%%%%%%%%%%%%%%%%%%%%%%%%%%%%%%%%%%%%%%%%%%%%%%%%%%%%%%%%%%%%%%%%%%%%%%%%%%%%%%%%%
% PREAMBLE
\documentclass[12pt]{article}

\usepackage{amsmath,amsfonts,bm,mathtools}
\usepackage[margin=0.21in]{geometry}

\pagenumbering{gobble}

% change bullet points to "-"
\renewcommand\labelitemi{\textbf-}

% box padding
\newcommand\padBox[1]{{\setlength\fboxsep{5pt}\boxed{#1}}}

% simplify creating binomial exponent:
% [3] parameters accpeted
% [2] is default value for first param (makes it optional)
% (#2 + #3)^#1 - what it does, puts 2nd + 3rd params in binomial, 1st is expon.
\newcommand{\negBi}[3][2]{(#2- #3)^{#1}}

% bold things in math or text mode using conditional logic
\newcommand*{\mB}[1]{\ifmmode\bm{#1}\else\textbf{#1}\fi}

\begin{document}
%%%%%%%%%%%%%%%%%%%%%%%%%%%%%%%%%%%%%%%%%%%%%%%%%%%%%%%%%%%%%%%%%%%%%%%%%%%%%%%%%
% SECTION: Measures of Spread
\section*{Measures of Spread}
% quick, easy enlarge equations
\begingroup
	\paragraph{Range, Variance and Stnadard Deviation}%
	\dotfill
	\large
	\begin{itemize}
		% create example datasets
		\item\textbf{Example Datasets:}
			\[
				\mB{X}=
				\left(10, 0, 10, 20, 30\right)
				\qquad\mB{and}\qquad
				\mB{Y}=
				\left(8, 9, 10, 11, 12\right)
			\]
		% get means
		\item\textbf{Population Mean:}
			\[
				\mB{X}=
				\mB{\mu}=\frac{\left(10 + 0 + 10 + 20 + 30\right)}{5}=
				\padBox{\mB{10}}
			\]
			\[
				\mB{Y}=
				\mB{\mu}=\frac{\left(8 + 9 + 10 + 11 + 12\right)}{5}=
				\padBox{\mB{10}}
			\]
		% get range
		\item\textbf{Population Range: }
			The difference between the high \& low datapoints in dataset.
			\[
				\mB{X}=
				(\enspace30 - (-10)\enspace) = \padBox{\mB{40}}
				\qquad\mB{and}\qquad
				\mB{Y}=
				(\enspace12 - 8\enspace) = \padBox{\mB{4}}
			\]
		% calculate variance
		\item\textbf{Variance: }
		Sum of squared differences of datapoints divided by total.
			\[
				\mB{X}=
				\mB{\sigma^{2}} = \frac{
					\negBi{-10}{10} +
					\negBi{0}{10} +
					\negBi{10}{10} +
					\negBi{20}{10} +
					\negBi{30}{10}
				}
				{5}
			\]
			\[=\qquad\frac{1000}{5}\qquad=\qquad\padBox{\mB{200^{u^2}}}\]
			\[
				\mB{Y}=
				\mB{\sigma^{2}} = \frac{
					\negBi{8}{10} +
					\negBi{9}{10} +
					\negBi{10}{10} +
					\negBi{11}{10} +
					\negBi{12}{10}
				}
			{5}
			\]
			\[=\qquad\frac{10}{5}\qquad=\qquad\padBox{\mB{2^{u^2}}}\]
	\end{itemize}

\paragraph{Summary:}%

We can clearly see which dataset has the greater variance or spread.
But the result $\sigma^{2}$ is a unit squared, meters, miles, etc.\
which brings us to finding the standard deviation $\sigma$.

	\begin{itemize}
		% get sd
		\item\textbf{Standard Deviation:}
			\[
				\mB{SD} = \sqrt{variance} = \sqrt{\sigma^{2}} = \sigma
				\quad\mB{then, }\quad
				\sigma = \sqrt{200} = 10\sqrt{2} \quad\mB{and}\quad \sigma = \sqrt{2}
			\]
	\end{itemize}
\endgroup
%%%%%%%%%%%%%%%%%%%%%%%%%%%%%%%%%%%%%%%%%%%%%%%%%%%%%%%%%%%%%%%%%%%%%%%%%%%%%%%%%
% SECTION: Sample Variance
\section*{Sample Variance}
\begingroup
	\paragraph{Sample Variance and Stnadard Deviation}%
	\dotfill
	\large
	\begin{itemize}
		% create example datasets
		\item\textbf{Sample Datasets:}
			\[
				\mB{X}=
				\left(10, 20, 30\right)
				\qquad\mB{and}\qquad
				\mB{Y}=
				\left(10, 11, 12\right)
			\]
		% sample formula
		\item\textbf{Sample Formula:}
			\[
				{\frac{\mB{\sum}_{i=1}^{n}(x_{i}-\bar{x}^{2})}{n- 1}}
			\]
	\end{itemize}

\paragraph{Explanation}%

Sum of all datapoints from $\mB{i=1}$, denoted with lowercase $\mB{n}$,
which is convention for the sample (uppercase $\mB{N}$ denoting the
population), then taking each datapoint $\mB{x_{i}}$, and from it
subtracting the sample mean $\mB{\bar{x}}$, then dividing it by one
less than the total datapoints $\mB{n - 1}$.

	\begin{itemize}
		% get means
		\item\textbf{Sample Mean:}
			\[
				\mB{X}=
				\mB{\bar{x}}=\frac{\left(10 + 20 + 30\right)}{2}=
				\padBox{\mB{30}}
			\]
			\[
				\mB{Y}=
				\mB{\bar{y}}=\frac{\left(10 + 11 + 12\right)}{2}=
				\padBox{\mB{16.5}}
			\]
		% get range
		\item\textbf{Sample Range: }
			The difference between the high \& low datapoints in dataset.
			\[
				\mB{X}=
				(\enspace30 - 10\enspace) = \padBox{\mB{20}}
				\qquad\mB{and}\qquad
				\mB{Y}=
				(\enspace12 - 10\enspace) = \padBox{\mB{2}}
			\]
		% calculate variance
		\item\textbf{Unbiased Sample Variance: }
		Sum of squared differences of datapoints divided by total minus one.
			\[\padBox{
				\mB{S_{n-1}^{2}}=\frac{\negBi{10}{30} +
					\negBi{20}{30} +
					\negBi{30}{30}
				}
				{2}
			}\]
			\[=\qquad\frac{500}{2}\qquad=\qquad\mB{250^{u^2}}\]
			\[\padBox{
				\mB{S_{n-1}^{2}}=\frac{\negBi{10}{16.5} +
					\negBi{11}{16.5} +
					\negBi{12}{16.5}
				}
			{2}
			}\]
			\[
				\approx\qquad\frac{92.75}{2}\qquad\approx\qquad\mB{46.38^{u^2}}
			\]
	\end{itemize}

\paragraph{Summary:}%

We can clearly see which dataset has the greater variance or spread.
But the result $\sigma^{2}$ is a unit squared, meters, miles, etc.\
which brings us to finding the standard deviation $\sigma$.

	\begin{itemize}
		% get sd
		\item\textbf{Sample Standard Deviation:}
			\[
				\mB{SSD}=\sqrt{variance}=\sqrt{S_{n-1}^{2}}= \mB{S}
			\]
			\[
				\quad\mB{then, }\quad
			\]
			\[
				\mB{S} = \sqrt{250} = 5\sqrt{10} \approx 15.81
				\quad\mB{and}\quad \mB{S} \approx \sqrt{46.38} \approx 6.81
			\]
	\end{itemize}

\paragraph{Note}%

Because the square root function is non-linear, this actually is not
a true unbiased estimate of the population standard deviation. We use
this sample variance to calculate the sample standard deviation but it
is NOT unbiased.

\endgroup
%%%%%%%%%%%%%%%%%%%%%%%%%%%%%%%%%%%%%%%%%%%%%%%%%%%%%%%%%%%%%%%%%%%%%%%%%%%%%%%%%
\end{document}

% \boldmath
