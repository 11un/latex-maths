%%%%%%%%%%%%%%%%%%%%%%%%%%%%%%%%%%%%%%%%%%%%%%%%%%%%%%%%%%%%%%%%%%%%%%%%%%%%%%%%%
% PREAMBLE
\documentclass[12pt]{article}

\usepackage{amsmath,amsfonts,mathtools,hyperref}
\hypersetup{colorlinks=true}
\usepackage[margin=0.21in]{geometry}

% padded box (answers)
\newcommand{\padBox}[1]{{\setlength\fboxsep{5pt}\boxed{#1}}}
% binomial exponent
\newcommand{\negBi}[3][2]{(#2- #3)^{#1}}
% large math
\newcommand*{\mL}[1]{\mbox{\Large$#1$}}
% huge math
\newcommand*{\mH}[1]{\mbox{\huge$#1$}}

\begin{document}
%%%%%%%%%%%%%%%%%%%%%%%%%%%%%%%%%%%%%%%%%%%%%%%%%%%%%%%%%%%%%%%%%%%%%%%%%%%%%%%%%
% SECTION: Normal Distribution Intuition
\section*{Normal Distribution Intuition}
\paragraph{Definition:}%

In probability theory, the normal (or Gaussian or Gauss or Laplace–Gauss) distribution is a very common continuous probability distribution. Normal distributions are important in statistics and are often used in the natural and social sciences to represent real-valued random variables whose distributions are not known. A random variable with a Gaussian distribution is said to be normally distributed and is called a normal deviate. \href{https://en.wikipedia.org/wiki/Normal_distribution}{Wikipedia}

\paragraph{Probability Density Function:}
\begin{equation}
	\mH{%
		p(x) = \frac{1}{a\sqrt{2\pi}}\exp
		\left(-\frac{\negBi{x}{\mu}}{2\sigma^2}\right)
	}
\end{equation}

%%%%%%%%%%%%%%%%%%%%%%%%%%%%%%%%%%%%%%%%%%%%%%%%%%%%%%%%%%%%%%%%%%%%%%%%%%%%%%%%
\end{document}
