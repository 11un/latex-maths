\documentclass[12pt]{article}
\usepackage[margin=6mm]{geometry}
\usepackage{amssymb, microtype, mathtools, booktabs}
% \usepackage[fleqn]{mathtools}

\begin{document}

\paragraph{Description} The properties of limits are intuitive but important for simplifying problems.

\paragraph{Setup}
\begingroup \large
\begin{align*}
  \lim_{x \to c} f(x) = L && \lim_{x \to c} g(x) = M
\end{align*}
\endgroup

\paragraph{Limit Properties}
\begin{flalign*}
  &\text{Sum Property:} && \lim_{x \to c} (f(x) + g(x)) &&
  = && L + M && \\ \\ % trailing "&&" for left flush
  &\text{Difference Property:} && \lim_{x \to c} (f(x) - g(x)) &&
  = && L - M \\ \\
  &\text{Product Property:} && \lim_{x \to c} (f(x) \times g(x)) &&
  = && L \times M \\ \\
  &\text{Constant Multiple Property:} && \lim_{x \to c} (k \times f(x)) &&
  = && k \times L \\ \\
  &\text{Quotient Property:} &&
  \frac{\lim\limits_{x \to c} (f(x))}{\lim\limits_{x \to c} (g(x))} &&
  = && L \div K \\ \\
    &\text{Exponent Property:} && \lim_{x \to c} {(f(x))}^{\frac{r}{s}} &&
  = && L^{\frac{r}{s}}
\end{flalign*}


\end{document}
