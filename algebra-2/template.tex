\documentclass[12pt]{article}
\usepackage[margin=6mm]{geometry}
\usepackage{amssymb, microtype, mathtools, booktabs}
\usepackage[table]{xcolor}

\begin{document}

\paragraph{Description} The properties of logarithms. The following
properties apply for any values of $M, N$ and $b$ for which each logarithm is
defined, which is $M,N > 0$ and $0 < b \ne 1$. Remember, in order for a
logarithm to be defined, the argument of the logarithm must be positive and
the base of the logarithm must be positive and not equal to 1.

{ % group

  \begin{center}

    \textbf{Logarithm Rules} \vspace{3mm} \\
    \large \def\arraystretch{2} % stretch row size
    \rowcolors{1}{white}{gray!10}

    \begin{tabular}{cccc}

      Product Rule & $\log_{b}(MN)$ & = & $\log_{b}(M) + \log_{b}(N)$ \\

      Quotient Rule & $\log_{b}(\frac{M}{N})$ &
                        = & $\log_{b}(M) - \log_{b}(N)$ \\

      Power Rule & $\log_{b}(M^{p})$ & = & $p\log_{b}(M)$ \\

      Change of Base Rule & $\log_{b}(M)$ &
                        = & $\frac{\log_{b}(M)}{\log_{a}(b)}$ \\

    \end{tabular} \\

  \end{center}

}

\end{document}
